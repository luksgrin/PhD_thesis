% English
\chapter*{\Large Stochastic dynamic modeling of evolving gene expression programs}

\vspace{-1cm}

\subsection*{\myself}

\subsection*{Abstract}
RNA viruses exemplify sophisticated genetic programs endowed with the capacity for change, offering a unique window into the interplay between intricate genomic design and dynamic potential. Building on a large-scale analysis of RNA virus genomic data, this work examines whether the accumulation of genetic modifications adheres to a constant-rate progression or exhibits more complex patterns. Central questions include how the distribution of genetic changes evolves over time, how distinct genetic profiles contribute to observed shifts, and whether these dynamics can be modeled by standard Poisson or Brownian processes. To address these questions, millions of publicly available sequences were processed, focusing on samples collected over an extended time period. Rigorous data filtering, multidimensional scaling, and temporal aggregation of mutational counts were employed to characterize both the mean and the variance in the number of substitutions relative to an early reference genome.

Findings reveal that while the average number of accumulated mutations increases linearly with time, variance in mutation counts follows a more intricate pattern than predicted by the classical molecular clock hypothesis. Specifically, marked accelerations of mutation rates occur when new variants, each carrying distinct genomic profiles, invade the population. Sub- or super-diffusive dynamics emerge in these variant-specific analyses, suggesting that the standard Brownian-motion model, commonly used to approximate neutral evolutionary processes, does not fully capture the underlying complexity. Instead, fractional Brownian motion with a time-dependent diffusion exponent better explains the observed discrepancies between mean and variance in mutation counts. This nuanced perspective is further supported by fluctuations in the ratio of nonsynonymous to synonymous substitutions (dN/dS), indicative of purifying selection with occasional adaptive events as new viral lineages spread.

Beyond refining long-standing molecular clock assumptions, this work contributes a novel analytical framework for understanding how population-wide dynamics, selection pressures, and transmission bottlenecks drive evolutionary trajectories in rapidly mutating RNA viruses. The integrative methodology, merging large-scale sequence data, variance-based metrics, and advanced modeling, can inform phylodynamic studies of other fast-evolving viruses. By exposing the limitations of purely Poissonian or Brownian models, these findings highlight the need for evolutionary frameworks that incorporate anomalous diffusion processes, profile-specific shifts in rate, and the broader ecological context of viral spread. Consequently, this work offers key insights into RNA virus evolution and underscores the importance of variance-based approaches in detecting critical deviations from conventional models, expanding our theoretical and practical understanding of viral adaptability and emergence.


\vfill

%%%%%%%%%%%%%%%%%%%%%%%%%%%%%%%%%%%%%%%%%%%%%%%%%%%%%%%%%%%%%%%%%%
% Spanish
\chapter*{\Large Modelado dinámico estocástico de programas de expresión génica evolutivos}

\vspace{-1cm}

\subsection*{\myself}

\subsection*{Resumen}
Los virus de ARN ejemplifican programas genéticos sofisticados dotados con capacidad de cambio, ofreciendo una ventana única a la interacción entre un intrincado diseño genómico y un potencial dinámico. Basándose en un análisis a gran escala de datos genómicos de virus de ARN, el presente trabajo examina si la acumulación de modificaciones genéticas se adhiere a una progresión de tasa constante o exhibe patrones más complejos. Las preguntas centrales incluyen cómo evoluciona la distribución de cambios genéticos a lo largo del tiempo, de qué manera perfiles genéticos distintos contribuyen a los cambios observados y si estas dinámicas pueden ser modeladas por procesos estándar de Poisson o Brownianos. Para abordar estas cuestiones, se procesaron millones de secuencias disponibles públicamente, concentrándose en muestras recolectadas durante un período temporal prolongado. Se aplicaron rigurosos filtros de datos, escalado multidimensional y agregación temporal de conteos mutacionales para caracterizar tanto la media como la varianza en el número de sustituciones con respecto a un genoma de referencia inicial.

Los hallazgos revelan que, mientras el número promedio de mutaciones acumuladas aumenta linealmente con el tiempo, la varianza en los conteos de mutaciones sigue un patrón más complejo del que predice la clásica hipótesis del reloj molecular. Específicamente, se observan aceleraciones marcadas en las tasas de mutación cuando nuevas variantes, cada una portadora de perfiles genómicos distintos, invaden la población. En estos análisis específicos de variantes emergen dinámicas sub- o super-difusivas, lo que sugiere que el modelo estándar de movimiento browniano, comúnmente utilizado para aproximar procesos evolutivos neutrales, no captura completamente la complejidad subyacente. En cambio, el movimiento fraccionario browniano con un exponente de difusión dependiente del tiempo explica de manera más adecuada las discrepancias observadas entre la media y la varianza en los conteos de mutaciones. Esta perspectiva matizada se ve respaldada además por las fluctuaciones en la proporción de sustituciones no sinónimas a sinónimas (dN/dS), indicativas de una selección purificadora con eventos adaptativos ocasionales a medida que se diseminan nuevas líneas virales.

Más allá de refinar las antiguas suposiciones del reloj molecular, este trabajo contribuye con un novedoso marco analítico para comprender cómo las dinámicas a nivel poblacional, las presiones selectivas y los cuellos de botella en la transmisión impulsan las trayectorias evolutivas en virus de ARN de rápida mutación. La metodología integradora, que fusiona datos de secuencias a gran escala, métricas basadas en la varianza y modelos avanzados, puede orientar estudios filodinámicos de otros virus de evolución acelerada. Al exponer las limitaciones de los modelos puramente Poissonianos o Brownianos, estos hallazgos resaltan la necesidad de marcos evolutivos que incorporen procesos de difusión anómala, cambios específicos en la tasa según el perfil y el contexto ecológico más amplio de la propagación viral. En consecuencia, el presente trabajo ofrece importantes aportes sobre la evolución de los virus de ARN y subraya la relevancia de los enfoques basados en la varianza para detectar desviaciones críticas de los modelos convencionales, ampliando así nuestra comprensión teórica y práctica de la adaptabilidad y emergencia viral.

\vfill

%%%%%%%%%%%%%%%%%%%%%%%%%%%%%%%%%%%%%%%%%%%%%%%%%%%%%%%%%%%%%%%%%%
% Valencian
\chapter*{\Large Modelatge dinàmic estocàstic de programes d'expressió gènica evolutius}

\vspace{-1cm}

\subsection*{\myself}

\subsection*{Resum}
Els virus d'ARN exemplifiquen programes genètics sofisticats dotats de la capacitat de canviar, oferint una finestra única sobre la interacció entre un disseny genòmic complex i un potencial dinàmic. Basant-se en una anàlisi a gran escala de dades genòmiques de virus d'ARN, aquest treball examina si l'acumulació de modificacions genètiques s'ajusta a una progressió de taxa constant o bé manifesta patrons més complexos. Les qüestions centrals inclouen com evoluciona la distribució dels canvis genètics al llarg del temps, com els diferents perfils genètics contribueixen als desplaçaments observats i si aquestes dinàmiques poden ser modelades per processos de Poisson o de Brown estàndard. Per abordar aquestes qüestions, es van processar milions de seqüències d'accés públic, amb un focus especial en les mostres recollides durant un període de temps prolongat. Es va dur a terme una rigorosa depuració de dades, escalat multidimensional i agregació temporal dels comptes de mutacions per caracteritzar tant la mitjana com la variància en el nombre de substitucions respecte a un genoma de referència inicial.

Els resultats revelen que, tot i que el nombre mitjà de mutacions acumulades augmenta de manera lineal amb el temps, la variància en els comptes de mutacions segueix un patró més intricada del que prediu la hipòtesi clàssica del rellotge molecular. Concretament, s'observen acceleracions marcades de les taxes de mutació quan noves variants, cadascuna amb perfils genòmics distintius, invadeixen la població. Dins d'aquestes anàlisis específiques per variant emergeixen dinàmiques sub- o super-difusives, la qual cosa suggereix que el model de moviment brownià estàndard, comunament utilitzat per aproximar processos evolutius neutres, no capta completament la complexitat subjacent. En lloc d'això, el moviment fraccionari brownià amb un exponent de difusió dependent del temps explica millor les discrepàncies observades entre la mitjana i la variància dels comptes de mutacions. Aquesta perspectiva matisada es veu reforçada per les fluctuacions en la relació de substitucions nonsinònimes a sinònimes (dN/dS), indicatives d'una selecció purificadora amb esdeveniments adaptatius ocasionals a mesura que es difonen noves línies vírals.

Més enllà de refinar les assumpcions del rellotge molecular de llarga data, aquest treball aporta un nou marc analític per entendre com les dinàmiques a nivell poblacional, les pressions selectives i els colls de botella en la transmissió impulsen les trajectòries evolutives en virus d'ARN que muten ràpidament. La metodologia integradora, que fusiona dades de seqüències a gran escala, mètriques basades en la variància i modelatge avançat, pot informar estudis filodinàmics d'altres virus d'alta evolució. En exposar les limitacions dels models purament poissonians o brownians, aquests resultats destaquen la necessitat de marcs evolutius que incorporen processos de difusió anòmala, canvis en la taxa específics de cada perfil i el context ecològic més ampli de la propagació viral. En conseqüència, la recerca ofereix coneixements clau sobre l'evolució dels virus d'ARN i subratlla la importància d'enfocs basats en la variància per detectar desviacions crítiques dels models convencionals, ampliant finalment la nostra comprensió teòrica i pràctica de l'adaptabilitat i l'aparició de nous virus.


\vfill