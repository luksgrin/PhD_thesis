\chapter{General Conclusions}

This PhD dissertation was intended to expand our knowledge on the evolutionary dynamics of rapidly mutating biological agents, with focus on SARS-CoV-2 as a model system for studying molecular evolution under natural conditions. By integrating mathematical modeling, stochastic processes, and large-scale genomic data analysis, we aimed to challenge traditional assumptions of the molecular clock and propose a more flexible, variant-dependent framework. The work further contributes to the development of more realistic models of viral evolution, offering insights into mutation rates, selective pressures, and diffusion patterns in genomic space. In particular, this thesis has reached the following main conclusions:

\begin{enumerate}
    \item RNA Virus evolution can follow variant-dependent, non-Brownian dynamics best described by a stochastic model incorporating fractional Brownian noise, where anomalous diffusion exponents quantify deviations from the classical molecular clock through subdiffusive or superdiffusive mutation rates.
    \item Variant-specific analyses reveal distinct mutation accumulation trajectories characterized by bursts in genetic variance during emergence and replacement events, yet converging toward asymptotic Poissonian behavior as captured by a reset-based dispersion index.
    \item Mutation accumulation in RNA viruses exhibits strong gene-level heterogeneity, deviating from the uniform patterns predicted by classical molecular clock models.
    \item Genes under functional constraint exhibit reduced variance and dispersion indicative of purifying selection, genes involved in host interaction undergo episodic adaptive bursts, marked by high dispersion indices and non-monotonic mutation dynamics.
    \item A top-down statistical analysis enables detection of localized evolutionary dynamics, revealing how gene-specific contributions shape genome-scale anomalous diffusion.
    \item Non-phylogenetic, population-level approaches offer scalable alternatives for modeling large viral datasets, underscoring the importance of implementing and distributing these novel models as open-source software to ensure broad accessibility and reproducibility.
\end{enumerate}

