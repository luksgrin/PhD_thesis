\chapter{General Discussion}

\begin{flushright}
    \textit{People think that mathematics is complicated.\\Mathematics is the simple bit, it's the stuff we can understand.\\It's cats that are complicated.}\\
    --- John H. Conway
\end{flushright}

\vspace{1cm}

This work set out to examine how stochastic dynamics shape both viral genome evolution and post-transcriptional regulatory programs. By uniting large-scale sequence analysis with mechanistic modeling, it challenged long-held assumptions of constant molecular clocks and purely transcription-focused gene regulatory frameworks. We explored RNA virus genomes as prime examples of dynamic molecular automata, investigating how mutations accumulate over time and whether these changes align with the classic neutral theory \cite{kimura1968,Zuckerkandl1965}. In addition, we briefly explored how bursty translational regulation as a source of noise that can create distinct gene expression states using synthetic gene circuits in bacteria as case study \cite{Elowitz2002,Dolcemascolo2022}. Together, these projects advanced a core hypothesis: biological systems often deviate from simple, uniform models, and statistical-based methods are essential for fully understanding their complexity. This work emphasizes that evolution and gene regulation are inherently stochastic processes, with constraints, and shifting contexts shaping their trajectories.

Key findings highlight the importance of anomalous diffusion in viral genome evolution. Using SARS-CoV-2 data, this work uncovers that mutation accumulation deviates markedly from a simple constant-rate (Poissonian) process. While the average number of substitutions increased roughly linearly with time (consistent with the classical molecular clock hypothesis in a broad sense), the variance in mutation counts was substantially higher or lower than expected at different periods. In particular, distinct variant-dependent patterns emerged: when new variants of concern arose and swept through the population, the evolutionary dynamics exhibited anomalous diffusion behavior. For example, the early Wuhan strain (Primal) and Alpha and Omicron variants showed subdiffusive dynamics -- mutations accumulated more slowly and incrementally, as if evolution was temporarily ``slowed down'' -- whereas the Delta variant showed superdiffusive behavior, with an accelerated pace of mutation accrual. This means that instead of a uniform random walk through sequence space, SARS-CoV-2 evolution alternated between periods of constrained, slower exploration and bursts of rapid change. Such findings are unprecedented in viral molecular evolution: they indicate that the virus' spread through sequence space cannot be captured by a single diffusion rate. Notably, these patterns corresponded to epidemiological transitions; bursts of genetic change coincided with the replacement of one variant by another (e.g., the rapid emergence of Alpha, then Delta, then Omicron). In these phases, the molecular clock ``ticks'' faster or slower depending on the variant, violating the assumption of a strictly constant rate. This result is strongly supported by our variance-based analysis and is further evidenced by fluctuations in the ratio of nonsynonymous to synonymous mutations (dN/dS) over time, which indicated predominantly purifying selection punctuated by episodes of adaptive evolution during variant sweeps. Contextualizing these findings, we note that the classical molecular clock concept (where genetic divergence grows linearly with time under neutral drift) has long been challenged by evidence of rate variation and overdispersion \cite{Ho2011}. Our observations of variant-specific rate shifts provide a concrete example of such overdispersion in a real-world dataset, echoing earlier suggestions that viral evolutionary rates can depend on the timescale and conditions of measurement \cite{Ho2011}. Traditional relaxed-clock phylogenetic models allow rates to vary across lineages \cite{Drummond2006}, but the fractional Brownian motion framework employed here offers a more explicit characterization of the anomalous diffusion we detected, capturing the memory and heterogeneity in substitution processes.

The notion of anomalous diffusion itself is well-established in other realms of biology: for instance, macromolecules in cells often diffuse subdiffusively due to crowding \cite{Weiss2004,Hofling2013}. Observing analogous subdiffusive and superdiffusive regimes in viral genome evolution is a novel insight of this thesis, effectively importing concepts from statistical physics into evolutionary biology. It suggests that viral populations exploring sequence space can behave like particles in a complex medium, sometimes experiencing constraints (perhaps due to fitness landscapes or transmission bottlenecks) and occasionally taking long jumps (e.g., after a fitness leap or immune escape mutation). 

Additionally, the brief exploration of bursty translational regulation as a source of noise at intracellular level ties into broader literature showing that noise and variability are fundamental in gene expression and can be harnessed by regulatory designs \cite{Paulsson2004}. Therefore, both intracellular processes and viral evolution deviate from simplistic models in predictable ways: anomalous diffusion and episodic rate changes emerge as unifying themes that connect our findings to foundational principles in biophysics and evolutionary theory \cite{kimura1968,Ho2011}. Despite these advances, there are important methodological and analytical limitations to acknowledge.

First, the sampling of viral genomic data is inherently biased. The sequences analyzed (e.g., SARS-CoV-2 genomes from global databases) are an opportunistic sample, skewed by factors such as surveillance intensity, geographic sampling disparities, and sequencing protocols over the pandemic's course. This means certain epochs or regions are over-represented while others (e.g., early spreading in low-surveillance areas) might be under-sampled, potentially influencing the observed mutation variance. Such bias is a common concern in phylodynamic studies and could lead to misestimation of diffusion parameters if, for instance, bursts of evolution coincided with periods of intensified sampling. Although we mitigated this by stringent data filtering and aggregation over time windows, a truly unbiased sample is unattainable and remains a caveat.

Second, there are assumptions in the modeling approach that constrain interpretation. The use of a fractional Brownian motion model with a single diffusion exponent exponent per viral variant, while capturing sub- or superdiffusive trends, simplifies the complex biological reality. In practice, multiple processes: continuous natural selection, changing transmission dynamics, and recombination events operate simultaneously, as we observed when diving deeper into a gene-by-gene analysis. Our model primarily attributes deviation from clock-like behavior to a diffusion exponent, effectively lumping together various causes into a single parameter. For example, purifying selection tends to remove deleterious mutations, which can lead to an apparent slowdown in divergence (subdiffusion) over time, whereas adaptive bursts (e.g., a new beneficial mutation sweeping) can cause sudden jumps (contributing to superdiffusion); our approach captures these in a phenomenological way but does not explicitly parameterize selection. Similarly, recombination (which is known to distort phylogenetic signals and clock estimates) was not explicitly modeled here \cite{Holmes1999}, although there are widely known cases of viral recombination in the scenario of a co-infection. While SARS-CoV-2's recombinant lineages were extremely infrequent during the studied period, any unnoticed recombination could violate model assumptions of a tree-like evolution and affect substitution counts.

Third, generalizability of the findings warrants caution. The viral evolution analysis was specific to SARS-CoV-2 during a pandemic scenario; RNA viruses with different life histories or mutation rates (e.g., measles virus, or a DNA virus like herpesvirus) might not exhibit the same diffusion characteristics. The anomalous diffusion observed might partly reflect the unique immune and epidemiological dynamics of COVID-19 (such as rapid global spread and strong variant sweeps). Applying our model to other pathogens may require adjustments, and one should be careful not to over-interpret the quantitative values (like the exact diffusion exponents) beyond this system.

Finally, there are limitations in the analytical methods themselves. Estimating evolutionary variance over time required binning sequences by sampling date, an approach that smooths over some detail; different bin sizes or outlier removal criteria could slightly change the measured degree of diffusion anomaly. We chose parameters based on robustness checks, but these choices involve trade-offs (e.g., temporal resolution vs. noise in variance estimates). 

Future research could explore these concepts across different viral families and more complex organisms. The anomalous diffusion framework could refine molecular epidemiological tools by identifying when a population transitions from one variant-dominated phase to another \cite{Ho2011}. Integrating temporal, spatial, and phenotypic data would improve our grasp of how adaptive lineages outcompete others under shifting selection pressures. Such efforts stand to enhance our predictive power, whether in forecasting viral spread or in engineering robust gene circuits. By tackling these open questions, this work's contributions can guide ongoing inquiries into the stochastic underpinnings of molecular evolution and improve both theoretical models and practical applications in biology.


\vfill

\pagebreak

\bibliographystyle{assets/rodrigostyle}
\bibliography{references/discussionReferences.bib}
